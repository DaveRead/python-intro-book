% LaTeX source for ``Introduction to Computer Science (Python Edition)''
% Copyright (c) 2021- David S. Read, All Rights Reserved

\chapter{Exploring Computer Science}

\section{What is Computer Science}
\index{Defining Computer Science}
\index{What is Computer Science}
\index{Computer Science}

Computer Science (CS) is a term used to describe a very broad field. Our everyday exposure to CS may leave us with a rather narrow view of computing. We often associate the concept of computing with hardware, such as cell phones ad laptop computers. However, the field of CS goes far beyond these everyday devices and the programs we often use to play games, interact on social media, or make online purchases.



\section{Impact of CS on Society}
\index{Impact of CS on Society}
\index{Society and CS}

CS has invaded our public and personal lives, often in ways we do not perceive. Its increasing presence is enabled by several factors including the low cost of computing devices, widespread access to the Internet, a prioritization of convience, and a desire on the part of corporations and governments to collect data for monitoring and monetization.

Forms and platforms: cell phones, cameras, GPS, tracking data, aggregation, risk assessment... 

Trading privacy for convience: crowdsourcing, "free" services

Bias


\section{Technology, the Internet of Things, and CS}
\index{Technology and CS}
\index{Internet of Things}
\index{IoT}

With the creation of an electronic computer, as discussed at the beginning of the book, governments, corporations, and individuals began to appreciate the value of a fast programmable calculator. Ongoing research found ways to reduce the size and energy consuption of these devices. Through the following decades, computers physically shrank while storing more data and running more calculations per second. 

This trend of being able to do more with less, identified by Moore as Moore's law, focuses on hardware. However, this observation isn't limited to computing hardware. Further, hardware is not the only area of advancement. We discussed the impact of algorithms and applications such as neural networks, that significantly support the application of CS to many fields.


\section{Intersection of CS and Career}
\index{Intersection of CS and Career}
\index{CS in the Workplace}

Computing's history began with a focus on banking and related fields, such as tax and payroll processing. From there it broadened into fields such as mechanical engineering, weather forecasting, and space exploration. Its uses generally delt with processing of numeric data and carrying out somewhat tedious but rather straightfoward calculations.

Once the computer was scaled down to a desktop size, in the early 1980s, there was a shift to provide programs that required less formal training to use. Two types of programs became successful in the workplace: word processors and spreadsheets.

Spreadsheets mimicked a common paper-based tool used by accountants. The electronic spreadsheet allowed the user to setup their financial calculations in a flexible and reusable manner. Those outside of finance began to find uses for these programs to store, sort, and report on all sorts of information.

Word processors allowed workers to create written documents in a streamlned manner. Prior to word processors, documentation and correspondence was typically hand-written or dictated and then typed by a secretary or assistant. The typed version was dificult to change and often had to be retyped for even small updates. Moving to a word processor allowed for people to skip the extra steps and maintain a typed document in a flexible and sharable manner.

From these humble beginnings the desktop computer became a standard office appliance, eventaully replacing typewriters and changing the role of a secretary or office assistant. This shift of role is now seen in many jobs as computers change the way work is performed across industries.

While the economy through much of the late 19th and early-to-mid 20th century was driven by manufacturing, a manufactring ecnomy. We have seen a shift in the last 40 years where companies derive value from information processing. The term "information economy" is used to describe this shift, and it impacts many careers. 

Note that manufacturing and information can co-exist in an economy. As information becomes more prevalent, the type of work changes. For example, while manufacturing in the early 20th century involved mostly human-operated machines, robots are now utilied for much of the direct manufacturing, while computers, information, are used to control them. Instead of operating levers on a drill press, a worker uses a computer to manipulates values that control the press.

Newer uses: retrieval of "space junk", exploration of Mars...

More everyday work examples: HVAC, robotic surgery, inventory management/just-in-time, trains...

Household: smart light bulbs, fitbit, Nest...

	
\section{CS-specific Careers}
\index{CS-specific Careers}
\index{Careers in CS}
\index{Jobs in CS}

Infrastructure/Operations/IT, Data security/CISO/pen tester/red team/blue team, software engineer, CIO/data management, QA, Data Science/Machine Learning/AI

CS in other fields: climate change, biodiversity, water management, ecology, education, modeling in many domains, waste management, logistics



\section{Glossary}

\begin{description}
	
	\item[Lorem Ipsum:]  Lorem ipsum dolor sit amet, consectetur adipiscing elit, sed do eiusmod tempor incididunt ut labore et dolore magna aliqua. Ut enim ad minim veniam, quis nostrud exercitation ullamco laboris nisi ut aliquip ex ea commodo consequat.
	\index{Lorem Ipsum}
	
\end{description}

\section{Exercises}

\begin{ex}
	Look up Moore's Law and explain the concept More was describing and its implications on computers, and other electronic devices. Where have you seen evidence of this?	
\end{ex}
\begin{ex}
	Research the term "information economy." Define and explain it in your own words. Think of a career or field of interest to you and suggest ways that computers might change how your work would be done.
\end{ex}

