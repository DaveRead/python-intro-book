% The contents of this file is 
% Copyright (c) 2015- David S. Read, All Rights Reserved
% Previous version, Copyright (c) 2009- Charles R. Severance, All Rights Reserved

\chapter{Preface}

Students learn in different ways - a seemingly obvious statement. Reviewing the literature on education, it appears as though the process of learning has been studied and written about for as long as people have desired to pass on knowledge. Various theories and models for learning have waxed and waned.

For example, one 7-style theory\footnote{\url{http://www.learndash.com/7-major-learning-styles-which-one-is-you/}} separates learners into visual, physical, aural, verbal, logical, social and solitary learners. An alternative style list, focused on senses, is provided by the VARK modalities\footnote{\url{http://vark-learn.com/introduction-to-vark/the-vark-modalities/}} of visual, aural, read/write and kinesthetic.

Regardless of the specific model chosen, there is basic agreement that \textit{different} people succeed by having \textit{different} ways of interacting with material in order to learn it. Further, any one person tends to benefit by having materials which present the class content using multiple modes.

This book's goal is to support the reading, verbal, and solitary modes of learning about computer science and Python programming. This compliments class presentations (visual, aural, logical, social modes) and hands-on lab work (write, kinesthetic, and physical modes). Put another way, this book does not provide a stand-alone or self-study platform for learning to program a computer.

But why create a book focused on Python programming at all? Python is a very mature programming language. There are many books that seek to teach the language and several of those also introduce programming and computer science concepts. My short answer to ``why?'' is simply, ``because I wanted one that aligned well with my approach to teaching computer science and Python.''

This book focuses on introducing the mechanics of programming using a third generation language (3GL), specifically Python. Most of the programming concepts apply to many other 3GLs such as C and Java. 

What the book does not expound upon\footnote{but arguably should} are object-oriented (OO) concepts which are vital to being an effective Python programmer. For my intro class, OO is discussed, reviewed and applied repeatedly in lectures and labs in the later half of the semester. I have punted in terms of attempting to document OO in this book. It remains a future to-do.

The foundation for this book comes from another open-source book. I took an on-line Python course taught by Dr. Charles Severance who was using a book he had written and released under an open license. I really liked his approach to the material and the writing style he used in the book.

I decided that having an open book which I could freely alter and extend would allow me to fill a learning mode gap that my slides, presentations, and lab assignments were unlikely to fill effectively. The result is this book. Combined with lectures, slides and labs, I hope it serves as a useful resource for students taking introductory computer science with me.

Finally, I would be remiss if I did not highlight a point I endeavor to reinforce in class, chiefly, \textit{you cannot learn to program simply by reading about programming or by reviewing completed programs}. Programming a computer is very similar to learning to play an instrument. 

Just as you learn to play the piano by repeatedly learning to play new pieces, you learn to program by going through the process of writing new programs. \textit{Repeatedly encountering novel requirements and having to figure out the programming constructs necessary to meet those requirement defines programming.} \textbf{Programming is a skill that has to be developed through doing, not observing.}

I wish you all the best as you begin your journey into computer science and programming. Whether you intend to pursue computer science as a major or you are just curious about computers and what they can do, I think you'll enjoy exploring logic and the basic process of creating software.

David S. Read\\
www.monead.com\\
Schenectady, NY, USA\\
August 5, 2016

\clearpage

\section*{Python for Informatics: Remixing an Open Book}

(Charles R. Severance)

\textit{This following text is from the prior version of this book (by Dr. Charles Severance) which was written as an introduction to Python programming.}

It is quite natural for academics who are continuously told to 
``publish or perish'' to want to always create something from scratch
that is their own fresh creation.   This book is an 
experiment in not starting from scratch, but instead ``remixing''
the book titled
\emph{Think Python: How to Think Like
a Computer Scientist}
written by Allen B. Downey, Jeff Elkner, and others.

In December of 2009, I was preparing to teach
{\bf SI502 - Networked Programming} at the University of Michigan
for the fifth semester in a row and decided it was time
to write a Python textbook that focused on exploring data
instead of understanding algorithms and abstractions.
My goal in SI502 is to teach people lifelong data handling 
skills using Python.  Few of my
students were planning to be professional 
computer programmers.  Instead, they
planned to be librarians, managers, lawyers, biologists, economists, etc., 
who happened to want to skillfully use technology in their chosen field.

I never seemed to find the perfect data-oriented Python
book for my course, so I set out 
to write just such a book.  Luckily at a faculty meeting three weeks
before I was about to start my new book from scratch over 
the holiday break, 
Dr. Atul Prakash showed me the \emph{Think Python} book which he had
used to teach his Python course that semester.  
It is a well-written Computer Science text with a focus on 
short, direct explanations and ease of learning.  

The overall book structure
has been changed to get to doing data analysis problems as quickly as
possible and have a series of running examples and exercises 
about data analysis from the very beginning.  

Chapters 2--10 are similar to the \emph{Think Python} book,
but there have been major changes. Number-oriented examples and
exercises have been replaced with data-oriented exercises.
Topics are presented in the order needed to build increasingly
sophisticated data analysis solutions. Some topics like {\tt try} and
{\tt except} are pulled forward and presented as part of the chapter
on conditionals.  Functions are given very light treatment until 
they are needed to handle program complexity rather than introduced 
as an early lesson in abstraction.  Nearly all user-defined functions
have been removed from the example code and exercises outside of Chapter 4.
The word ``recursion''\footnote{Except, of course, for this line.}
does not appear in the book at all.

In chapters 1 and 11--16, all of the material is brand new, focusing
on real-world uses and simple examples of Python for data analysis 
including regular expressions for searching and parsing, 
automating tasks on your computer, retrieving data across 
the network, scraping web pages for data, 
using web services, parsing XML and JSON data, and creating 
and using databases using Structured Query Language.

The ultimate goal of all of these changes is a shift from a 
Computer Science to an Informatics
focus is to only include topics into a first technology 
class that can be useful even if one chooses not to 
become a professional programmer.

Students who find this book interesting and want to further explore
should look at Allen B. Downey's \emph{Think Python} book.  Because there
is a lot of overlap between the two books,
students will quickly pick up skills in the additional
areas of technical programming and algorithmic thinking 
that are covered in \emph{Think Python}.
And given that the books have a similar writing style, they should be 
able to move quickly through \emph{Think Python} with a minimum of effort.

\index{Creative Commons License}
\index{CC-BY-SA}
\index{BY-SA}
As the copyright holder of \emph{Think Python},
Allen has given me permission to change the book's license 
on the material from his book that remains in this book
from the
GNU Free Documentation License 
to the more recent
Creative Commons Attribution --- Share Alike
license.
This follows a general shift in open documentation licenses moving 
from the GFDL to the CC-BY-SA (e.g., Wikipedia).
Using the CC-BY-SA license maintains the book's 
strong copyleft tradition while making it even more straightforward 
for new authors to reuse this material as they see fit.

I feel that this book serves an example of why open 
materials are so important to the future of education,
and want to thank Allen B. Downey and Cambridge University
Press for their forward-looking decision to make the book available
under an open copyright.   I hope they are pleased with the 
results of my efforts and I hope that you the reader are pleased with
\emph{our} collective efforts.

I would like to thank Allen B. Downey and Lauren Cowles for their help,
patience, and guidance in dealing with and resolving the copyright 
issues around this book.

Charles Severance\\
www.dr-chuck.com\\
Ann Arbor, MI, USA\\
September 9, 2013

Charles Severance is a 
Clinical Associate Professor 
at the University of Michigan School of Information.

\clearemptydoublepage
